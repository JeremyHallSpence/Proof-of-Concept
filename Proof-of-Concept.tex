%Preamble

\documentclass[twoside, 11pt]{article}

%Start of  Macquarie University formatting requirements

%Sets margins
\usepackage
[	
a4paper,
top=1.5cm,
bottom=1.5cm,
left=1.5cm,
right=1.5cm,
bindingoffset=2cm,
heightrounded,
includehead,
includefoot
]{geometry}

%Sets line spacing
\usepackage{setspace}
\onehalfspacing

%End of Macquarie University formatting requirements

%Sets paragraph spacing
\setlength{\parskip}{1em}

%Removes widow and orphan lines.
\widowpenalty10000
\clubpenalty10000


%Title information
\title {Analytic Report - Ending Footbinding and Infibulation}
\date {5/5/2017}
\author {Jeremy Hall Spence \\ Department of Media, Music, Communication and Cultural Studies \\ Macquarie University}


%Document start

\begin{document}


%Title page%

\pagenumbering{gobble}

\maketitle

\cleardoublepage

%Table of contents

\tableofcontents

\cleardoublepage

%Abstract

\begin{abstract}

\normalsize

Lorem ipsum dolor sit amet, consectetur adipiscing elit. Aliquam interdum ultricies vestibulum. Cras non felis eros. Curabitur pretium erat dapibus maximus posuere. Aliquam enim neque, fermentum eget auctor at, blandit sit amet velit. Fusce scelerisque dolor in ex mollis vulputate. Donec consequat facilisis lorem euismod tristique. Quisque et ex ut tellus gravida sagittis vitae suscipit nulla. Curabitur ornare ante vitae ipsum pretium, in posuere elit ullamcorper. Donec non risus ligula.

Duis porta dui leo, nec faucibus ex tincidunt vel. Nullam sit amet egestas magna, vel maximus ex. Curabitur enim leo, rhoncus vel est sit amet, convallis faucibus ex. Nullam diam nisi, rhoncus scelerisque urna quis, vehicula finibus nulla. Donec in fermentum arcu. Curabitur fermentum felis purus, id sagittis lorem viverra in. Maecenas hendrerit orci id pretium eleifend. Pellentesque faucibus, dui vitae cursus malesuada, dui nibh congue orci, hendrerit efficitur nunc massa a elit. Maecenas ultricies eros enim, sit amet fermentum nunc molestie ut. Cras maximus tellus tellus. Morbi velit ante, molestie imperdiet fringilla vel, pulvinar ut orci. Phasellus viverra diam sed nulla lobortis interdum.

Vestibulum et urna et erat tristique sagittis. Ut eget pulvinar nibh. Vivamus rutrum arcu sed fringilla suscipit. Fusce euismod, est quis tristique bibendum, erat diam luctus.

\end{abstract}

\pagenumbering{arabic}

%Start of body text

\section*{Introduction}
\addcontentsline{toc}{section}{Introduction}
	Gerry Mackie's \textit{Ending Footbinding and Infibulation}, \cite{Mackie1996} concerns the similar practises of footbinding in China and infibulation, or female genital mutilation, in north-east Africa. In it, he argues that the practise of footbinding in China lasted for 1,000 years but ended in a single generation, and analyses the causes of the end of the tradition. He then recommends that by employing the same practises in north-east Africa, the practise of infibulation would be ended in the same manner. However, this essay is largely unconcerned with this argument, and is instead an analysis of the three main paradigms employed by the author in his argument, modernisation theory, rational choice theory, and structural functionalism. 

\section*{Modernisation Theory}
\addcontentsline{toc}{section}{Modernisation Theory}
	The basic idea of modernisation theory is that societies develop through a number of stages, the final stage of which is democratisation. \cite[p.158]{Przeworski1997} Developing countries are considered to be at a 'pre-modern' stage, but as they go through the stages of modernisation, they become more like the western world in regards to culture, economy and politics. The modernisation process often targets cultural traditions which are incompatible with the cultural tenants of the modernised west. 

	In Mackie's text, the traditions of infibulation and footbinding are the target of modernisation, and the author both describes the historical process of modernisation in early 1900s China, and recommends a current modernisation of north-east Africa. In the case of the historical modernisation of China, this is particularly conspicuous, as Mackie attributes the end of footbinding to three causes, a pledge society formed by Protestant missionaries, the national Natural Foot Society, organised by western women to propagandise the disadvantages of footbinding, and international disapproval conveyed by the society.\cite[p.1001]{Mackie1996} All of these were originally created by westerners and aimed at the non-Christian elite, encouraging them to abandon their traditional practises of footbinding, which the west disapproved of. It was only after this intervention that the public opinion in China at the time began to shift to an opposition to footbinding, leading to a total end to the practise by 1919. 

	One aspect that makes the author's use of this theory different, is that he then uses the historical process of the ending of footbinding in China, and applies it to modern-day north-east Africa with the custom of infibulation, saying that ``The sudden end of footbinding in response to abolitionist campaigns supports the convention hypothesis and predicts that equivalent African campaigns could help end FGM.'' \cite[p.1015]{Mackie1996} While this is not a typical approach, Mackie's use of modernisation theory is secondary to the other paradigms he uses, particularly rationalisation theory, and he is clearly more interested in how the three factors that ended footbinding effected individual decisions, rather than how the west influenced those groups. This is demonstrated when he mentions the work of another theorist taking a modernisation approach, who argues that footbinding declined due to cheaper products of modern industry displacing the products of female household labour. Mackie acknowledges this as true, but that there is evidence in her work of the indirect effects of the reforming associations on individuals. \cite[p.1014]{Mackie1996}

\section*{Rational Choice Theory}
\addcontentsline{toc}{section}{Rational Choice Theory}
	Rational choice theory is a school of thought used to model decision making, based on the assumption that all humans will act rationally in regards to their preferences. Furthermore, it assumes that all rational agents will calculate the costs and benefits of every action before deciding what to do, and that therefore all complex social phenomena can be explained in terms of individuals making rational decisions. \cite[p.126-127]{Scott1999} Two variants of Rational choice theory are \textit{decision theory}, which concerns cost-benefit calculations that are made in isolation, or without any other agents interfering, and \textit{game theory}, which analyses choices made when another agent is present in the equation, forcing an agent to consider what the other agent will do. The benefits of rational choice theory are that it eliminates much of the uncertainty around human behaviour, in order to allow the analysis of more complex social systems. \cite[pp.131-133]{Scott1999}

	The core argument of \textit{Ending Footbinding and Infibulation} is entirely based on rational choice theory, and games theory in particular, the principles of which the author outlines in his section on conventions. \cite[pp.1005-1007]{Mackie1996} His main argument is that the practises of footbinding in China and infibulation in north-east Africa persisted, and continued to persist, due to a system of inferior conventions, leading to sub-optimal outcomes in game simulations. For example, though both men and women would be better off marrying without mutilations, which is the superior convention, due to the belief that unmutilated women are unfaithful, men would prefer a mutilated woman, despite it not being the best outcome. In turn, women fear that if they are not mutilated, they will not be able to find a husband due to this belief, and therefore that mutilation, though an inferior outcome, is safer than the alternative. \cite[p.1008]{Mackie1996} So, while either agent would prefer non-mutilation, because they are forced to consider what the other agent will do, the rational decision is for mutilation, thus leading to the reinforcement of the inferior convention. Furthermore, he also argues that by the introduction of education on the physiological effects, the deploring of the bad health consequences by the international community, and the creation of pledge associations not letting sons marry mutilated women, all drawn from the successful Chinese movement to end footbinding, there would be enough of an impact on the cost-benefit decisions made by the actors that they could escape this self-reinforcing inferior convention. \cite[p.1012]{Mackie1996} He illustrates this with a Schelling coordination diagram, used for visualising coordination games in games theory.

	One peculiar aspect of Mackie's use of rational choice theory, is that instead of just explaining the social phenomena of a society, he is examining two social phenomena in two different societies, and applying the successful results of one to the other. In doing so, he is arguing that these two phenomena are based on enough of the same factors in regards to the cost-benefit analysis, that the same rational choices will be made by agents in both societies, despite their differences. It is likely that he deployed rational choice theory in this manner due to a lack of current information available, as he says that ``the technique of pledge associations is not reported in the FGM literature.'' \cite[p.1016]{Mackie1996} Therefore, he would have had to draw from previous similar practises, in this case footbinding, in order to examine what the effects of such a social change would be on rational agents. However, in doing so the author has created some limitations within his work, the main limitation being that although the phenomena have similar social factors, as is outlined extensively in the text, culturally the practising areas are very different, and his use of rational choice theory fails to account for those differences. One such difference is that while the education campaign in China emphasised that China was the only country in the world to practise foot binding, \cite[p.1010]{Mackie1996} there are several countries that practise infibulation, which would likely diminish the effects of the campaign. This would in-turn alter the outcome of the cost-benefit analysis made by individuals, making them more likely to remain with the inferior convention.

\section*{Structural Functionalism}
\addcontentsline{toc}{section}{Structural Functionalism}
	Structural functionalism, in its most basic form, is a form of analysis based on the idea that all social phenomenon can be examined by its role in a broader society. All social structures and institutions are viewed as interlocking and interdependent parts, which support and stabilise each other, and are defined by their function to the whole. In turn, a society can be viewed as a large system of these parts, with the functioning of society based on the functioning of its systems. Typically, structural functionalism has very little regard for the individual people in its analysis, as the group acts very differently than a person. 

	The use of structural functionalism in the text is certainly not as prominent as rational choice theory, and Mackie avoids using any of the most notable keywords of the paradigm, such as systems or functions. However, the core of the paradigm still exists within the text, particularly in the introductory sections on footbinding and infibulation. In each, the author dedicates a section to explaining the functions they played in society, and mentions that both phenomena's functions were to secure a proper marriage and ensure fidelity. \cite[pp.1001-1002, 1004]{Mackie1996} He also mentions that a common explanation given of infibulation, is that ``the practise functions to promote the solidarity of the group.'' \cite[p.1004]{Mackie1996} These explanations can be seen as having the hallmarks of the structural functionalist approach, as they examine footbinding and infibulation by the function they play in the wider society, be that marriage or social solidarity. However, structural functionalism can also be read in the author's recommendation on how to create a movement to end infibulation, if we view the solution as a system with three main parts: The pledge associations, the international judgement and the education in regards to the physiological effects. All of these are interlocking and integral parts of the system, as Mackie argues that without any one of these, the anti-mutilation movement would not be able to reach a tipping point, \cite[p.1012]{Mackie1996} and are examined by the function they serve in the greater system, in the section on escaping inferior conventions.\cite[pp.1011-1014]{Mackie1996}

	The most interesting aspect of Mackie's use of structural functionalism however, is that he combines it with his use of rational choice theory, as these theories seem almost diametrically opposed. Rational choice theory, as discussed above, explains all social phenomena in terms of individuals making choices, whereas structural functionalism examines social phenomena as part of a greater system, with little influence from the individual. However, structural functionalism was deployed in this manner in order to support the core argument based on rational choice theory, by outlining the individual's attitude towards the practice. As Mackie states: ``as soon as women believed that men would not marry an unmutilated woman, and men believed an unmutilated woman would not be faithful\ldots a self-enforcing convention was locked in'' \cite[p.1008]{Mackie1996} By demonstrating the greater role in society that footbinding and infibulation plays, Mackie highlights why it is rational for parents to continue these traditions on their children, as they fear the exclusion of their daughters from marriage. Structural functionalism is limited in its  use here, as Mackie makes note of, saying that ``functionalism explains everything but change.'' \cite[p.1010]{Mackie1996} Structural functionalism can be employed to explain why these conventions of mutilation have been so enduring, but rational choice theory must be deployed to explain how the cessation of these practises occur. 

\section*{Conclusion}
\addcontentsline{toc}{section}{Conclusion}
	In conclusion, while rational choice theory is the main paradigm employed by Mackie in the text,  both modernisation theory and structural functionalism are used to support it. The historical aspects of the end of footbinding in China are covered by modernisation theory, but once Mackie has determined the three main aspects that influenced social change, he analyses their effects through rational choice theory. Similarly, Mackie employs the techniques of structural functionalism to examine how footbinding and infibulation factored into their respective societies, and could thereby assess the effects of this societal pressure on the choices made by individuals. So, while rational choice theory constitutes the main argument of Mackie's work, it is also heavily reliant on modernisation theory and structural functionalism. These allow Mackie to assess all the factors of the cost-benefit analysis being made by individuals, and in turn create an accurate simulation of the coordination game involved in ending such practises. While some of these paradigms, particularly structural functionalism and rational choice theory, seem incompatible in their scope, the author uses them in a coordinated manner, to  analyse different aspects of the same social phenomena and synthesise the data he needs.
\newpage

%References
\addcontentsline{toc}{section}{References}
\bibliographystyle{apalike}
\bibliography{proofofconcept}

\end{document}
